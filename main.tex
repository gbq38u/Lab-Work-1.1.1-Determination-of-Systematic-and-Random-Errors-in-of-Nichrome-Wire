\documentclass[a4paper, 10pt]{article}%Тип документа

%Настройка  полей
\usepackage[left=3cm,right=3cm]{geometry}

%Русский язык
\usepackage[T2A]{fontenc} %кодировка
\usepackage[utf8]{inputenc} %кодировка исходного кода
\usepackage[english,russian]{babel} %локализация и переносы



%Вставка картинок
\usepackage{graphicx}
\graphicspath{{pictures/}}
\DeclareGraphicsExtensions{.pdf,.png,.jpg}

%Графики
\usepackage[dvips]{graphicx}

%Математика
\usepackage{amsmath, amsfonts, amssymb, amsthm, mathtools}

%Заголовок
\title{ Отчет о выполнении лабораторной работы \\ 
\textbf{Определение систематических и случайных погрешностей при измерении удельного сопротивления нихромовой проволоки 1.1.1}}
\author{Г.А. Кузин}
\date{ Группа Б01-501 г. Долгопрудный, ФРКТ МФТИ, 04.09.2025 \\}
\begin{document}
\maketitle
\newpage
\pagebreak 
\textbf {Аннотация:}\\
В работе определили удельное сопротивление нихрома тремя разными способами. Также для него определены вольтамперные характеристики для схем, использовавшихся в процессе работы. Получены 3 значения удельного сопротивления нихрома, самое точное из которых имеет точность  примерно $5\%$. Полученные значения совпали между собой и с табличным значением в пределах погрешности. \\
\newpage
\pagebreak
\textbf {Введение:}\\
При разработке и производстве электронагревательных элементов, точных измерительных приборов и систем автоматики критически важным параметром является удельное электрическое сопротивление материалов. Одним из ключевых явлений, позволяющих анализировать поведение проводников в электрической цепи, является преобразование электрической энергии в тепловую, которое определяется изменением сопротивления материала под воздействием силы тока и приложенного напряжения. Точное определение удельного сопротивления позволяет оптимизировать процессы, связанные с проектированием и калибровкой электрооборудования.
Одним из методов определения удельного сопротивления проводника является его прямое измерение с использованием мостовых схем, таких как мост Уитстона, обеспечивающих высокую точность. На основе полученных данных можно также косвенно рассчитать удельное сопротивление, применяя закон Ома для однородного участка цепи, что позволяет оценить применимость различных методик для конкретных условий.
Целью данной работы являлось экспериментальное определение удельного электрического сопротивления нихрома различными методами (прямым и с использованием моста Уитстона) в заданном диапазоне токов, сравнение полученных результатов с табличными данными и оценка применимости каждого из методов для достижения максимальной точности. Точное определение удельного сопротивления нихрома имеет важное практическое значение, так как позволяет судить о химической однородности и качестве сплава, что напрямую влияет на надежность и срок службы электротехнических устройств.\\
\newpage
\pagebreak
\textbf {Методика:}  \\
В начале работы необходимо собрать установку, позволяющую определить сопротивление проволки максимально точным способом (в моей работе были испойзованы схемы показанные на рисунке вприложении).
Для определения удельного сопротивления металла можно использовать проволку из этого материала, площадь сечения которой рассчитывается по формуле: 
\[ S = \dfrac{ \pi d^2 }{4}\]\\
где S - площадь сечения проволки, d - диаметр проволки\\
Далее необходимо измерить вольтамперую характеристику этой проволки для каждой из двух схем (см. рис. 1) и для разных длин проволки. Используя данные измерений вольтамперной характеристики можно найти сопротивление нихромовой проволки из закона Ома:
\[ Rпр = \dfrac{V}{I} \]
где Rпр - сопротивление проволки, V  - напряжение, I - сила тока.
Но из-за влияния внутреннего сопротивления прибора Rпр отличается от искомого Rиск. Связь Rпр и Rиск:
\[ R_\text{иск} = R_\text{пр} + \dfrac{R_\text{пр}^2}{R_\text{v}}\]
где Rv - сопротивление вольтметра
Далее можно произвести измерения сопротивления нихромовой проволки для разных длин проволки с помощью моста Уинстена. Для нахождения удельного сопротивления металла можно воспользоваться формулой:
\[\rho = R \frac{\pi d^2}{4l} \]
где R \bfseries -- \mdseries сопротивление проволоки, d \bfseries -- \mdseries её диаметр, l \bfseries -- \mdseries её длина.\\
\newpage
\pagebreak 





\textbf {Результаты и их обсуждение:}\\ 
Измерения диаметра проводились с помощью штангенциркуля ($d_1$, табл. 1) и микрометра ($d_2$, табл. 2) на 10 различных участках. При измерении диаметра проволоки штангенциркулем случайная погрешность отсутствует. Следовательно, точность результата определяется только точностью штангенциркуля $  d_1 = (0,4 \pm 0,1)$ мм. При измерении микрометром есть как систематическая, так и случайная ошибка (рассчет в приложении)  $  d_2 = (0,36 \pm 0,1)$ мм. Т.к. погрешность микрометра на порядок меньше погрешности штангенциркуля, для расчета площади поперечного сечения проволоки использовалось значение, полученное с помощью микрометра. Для нахождения вольтамперной характеристики были собраны две различные схемы представленные на рис. 1 а) б) (см. приложение).
Площадь поперечного сечения проволоки получилась (расчёт в приложении):\\
$S = (9{,}0 \pm 0{,}5)\cdot 10^{-2}\,\text{мм}^2$, т.е. площадь поперечного сечения
проволоки определена с точностью $5\%$.\\

Получены зависимости силы тока от напряжения для этих трёх отрезков проволоки разной длины, представленные на графике. 
\centering {\includegraphics[width= 150 mm,heigth=300 mm]{графика.png}}
\item (рис. 3) Зависимость силы тока от напряжения для схемы а)
\item Так как зависимость линейная, то коэфиценты наклона прямой рассчитые по методу наименьших квадратов для 50см - 1,95, для 30см - 3,42, для 20см - 4,88.
\item По формулам (см. в приложении) найдём среднеквадратичную случайную ошибку и возможную систематическую погрешность $R_\text{ср}$. Собираем схему рис. 1б. Снова проводим опыты для $l_\text{1}, l_\text{2}, l_\text{3}$ при возрастающих и убывающих значениях тока. Измерения записываем в табл. 6, 7, 8.(см. приложение). Проведя аналогичные расчёты, заносим средние значения величин в таблицу 10, считаем . Получены зависимости силы тока от напряжения для этих трёх отрезков проволоки разной длины, представленные на графике. \\
\centering {\includegraphics[width= 150 mm,heigth=300 mm]{графика.png}}
\item (рис. 4) Зависимость силы тока от напряжения для схемы б)
\item Так как зависимость линейная, то коэфиценты наклона прямой рассчитые по методу наименьших квадратов для 50см - 1,83, для 30см - 2,86, для 20см - 4,4.
\item На графиках не показаны погрешности в виде "крестов" так как погрешности измерений получились слишком палыми и их не будет видно на графиках(они укладываются в пределы отмеченных точек).
\item При помощи моста Уитстона (измерительный мост постоянного тока P4833) измеряем сопротивления (обозн. $R_\text{0}$) и также заносим в таблицу 11(см. приложение). Анализируя полученные значения (см. табл. 11) видим, что значения, полученные при использовании схемы а) ($\pm$ 2,5 \text{\%}) ближе к значениям, полученным при помощи моста Уитстона, чем для схемы б) ($\pm$ 8 \text{\%}): это значит, что схема а) точнее определяет небольшие сопротивления, чем схема б), что подтверждает теоретические расчёты. Находим удельное сопротивление и его погрешность для каждой из длин проволоки и заносим эти значения в табл.12 (см. приложение). Получаем:$\rho = (0,98\pm0,05) \dfrac{\text{Ом} \cdot \text{мм}^2}{\text{м}}$. \\

\newpage
\pagebreak 
\textbf{Выводы:} \\
Экспериментально получено, что удельное сопротивление нихрома составляет $1.05 \pm 0.05 \dfrac{\text{Ом} \cdot \text{мм}^2}{\text{м}}$. Полученное значение сравниваем с табличными данными: в справочнике (Таблицы физических величин: Справочник И.К. Кикоин, 1976 г.) для удельного сопротивления нихрома при 20 $^\circ$C указан диапазон значений от 0.97 до 1.12 $\dfrac{\text{Ом} \cdot \text{мм}^2}{\text{м}}$ в зависимости от марки сплава. Экспериментальное значение попадает в этот диапазон, что свидетельствует о хорошем качестве исследуемого образца проволоки. Основной вклад в погрешность измерений вносит определение диаметра проволоки, составляющее примерно $5\%$. Это связано с трудностями точного измерения малых линейных размеров микрометром и возможной неоднородностью толщины проволоки по длине. Измерения длины проводника и электрических параметров цепи были выполнены с существенно меньшей относительной погрешностью. Таким образом, применение обоих методов измерения (прямого и с использованием моста Уитстона) позволило получить достоверное значение удельного сопротивления, адекватно описывающее свойства конкретного образца нихромовой проволоки.\\
\pagebreak		
\textbf {Приложение:}\\
Формула сопротивления проволки:
\[ \hspace{30 mm}   \rho = R \frac{\pi d^2}{4l},		\eqno(1) \]
где R \bfseries -- \mdseries сопротивление проволоки, d \bfseries -- \mdseries её диаметр, l \bfseries -- \mdseries её длина.
\parindent Согласно закону Ома напряжение V и ток I в образце связаны соотношением
\[ \hspace{30mm} V = R I \eqno(2) \]
Рассчет погрешностей микрометра:
\[ \sigma_{\text{сист}}  = 0,01 \text{ мм, }
\sigma_{\text{сл}}  = \dfrac{1}{N} \cdot \sqrt{\sum_{i = 1}^N (d - \overline{d})^2} = \frac{1}{10}\sqrt{2,4\cdot10^{-4}} \approx 1,6\cdot10^{-3} \text{ мм}\]
\[ \sigma =  \sqrt{\sigma_{\text{сист}}^2 + \sigma_{\text{сл}}^2} \approx 0,01 \text{ мм}\]

Теоретически, надо измерять способом показанным на рис. 1a, так как: для схемы на рисунке 1а: $R_{\text{пр}}$/$R_{\text{V}}$ = 5/500 = 0,01, т.е. 1\%; а для схемы на рисунке 1б: $R_{\text{A}}$/$R_{\text{пр}}$ = 1/5 = 0,2, т.е. 20\%. То есть при измерении относительно небольших сопротивлений меньшую ошибку даёт схема рис. 1а.
\vspace {5 mm}
\[ S = \dfrac{ \pi d^2 }{4} = \dfrac{3,14 \cdot (0,34 \,\text{})^2}{4} \approx 0,09 \,\text{мм}^2\]
Найдём погрешность площади поперечного сечения проволоки:
\[ \left\sigma_{s} =\dfrac{2 \text{ }\sigma_{d_2}}{d_2} \cdot S = \dfrac{2 \text{ } \cdot 0,01}{0,34} \cdot 0,09  \approx 5 \cdot 10^{-3} \text{ мм}^2\]
Формула углового коэффицента
\[ R = \dfrac{ \left\langle VI \right\rangle }{ \left\langle I^2 \right\rangle } \]

Формулы среднеквадратичной случайной ошибки и возможной систематической погрешности
\[ \sigma^{\text{случ}}_{R_{\text{ср}}} = \dfrac{1}{\sqrt{N}}\cdot \sqrt{\dfrac{\left\langle V^2 \right\rangle }{\left\langle I^2 \right\rangle} - R^{2}_{\text{ср}}}\] 
\[ \sigma^{\text{сист}}_{R_{\text{ср}}} = R_{\text{ср}}\sqrt{\left( \dfrac{\sigma_{V}}{V} \right) ^{2} + \left( \dfrac{\sigma_{I}}{I} \right) ^{2} } \]
\[ \sigma_R =  \sqrt{\sigma_{\text{сист}}^2 + \sigma_{\text{сл}}^2} \]
\[ R_\text{cр} = \dfrac{V}{I}\eqno(2) \]
\[ R_\text{пр} = R_\text{ср} + \dfrac{R_\text{ср}^2}{R_\text{V}} \text{ - для схемы а)}\]\
\[ R_\text{пр} = R_\text{cр} \cdot (1 - \dfrac{R_\text{ср}}{(R_\text{A} + R_\text{ср})}) \text{ - для схемы б)}\]
Формула удельного сопротивления проволки и его погрешность:
\[\rho = \dfrac{R \cdot S}{l}\]
\[ \sigma_{\rho} = \rho \sqrt{\left( \dfrac{\sigma_{R}}{R} \right) ^{2} + \left( \dfrac{\sigma_{l}}{l} \right) ^{2} + \left( \dfrac{\sigma_{S}}{S} \right) ^{2} } \]
\begin{figure}[htbp]
    \centering
    
    \includegraphics[width=0.7\linewidth]{Рис.1.png} 
    \caption{ Схемы для измерения сопротивления при помощи амперметра и вольтметра}
    \label{fig:my_png}
\end{figure}
\newpage
\begin{table}
	\caption{Результаты измерения диаметра проволоки}
	\begin{tabular}{|r|c|c|c|c|c|c|c|c|c|c|c|}
	\hline
& 1 & 2 & 3 & 4 & 5 & 6 & 7 & 8 & 9 & 10\\
\hline
$d_1$, мм & 0,3 & 0,4 & 0,4 & 0,4 & 0,4 & 0,3 & 0,3 & 0,4 & 0,4 & 0,4 \\
\hline
$d_2$, мм & 0,35 & 0,33 & 0,34 & 0,33 & 0,34 & 0,34 & 0,35 & 0,34 & 0,36 & 0,34 \\
\hline
\multicolumn{1}{|r|}{} & \multicolumn{5}{c}{ \( \overline{d_{1}} = 0,4 мм \) мм}  & \multicolumn{5}{c|}{ \( \overline{d_{2}} = 0,34 мм \) мм}\\
\hline
\end{tabular}
\end{table}

\begin{table}
\caption{Основные характеристики амперметра и вольтметра}
\begin{tabular}{|p{3cm}|p{5cm}|p{5cm}|}
\hline
& Вольтметр & Амперметр \\
\hline
Система & Магнитоэлектрическая & Электромагнитная\\
\hline
Класс точности& 0,5& 0,5\\
\hline
Предел измерений $ x_{n} $& 0,6 В & 0,15 А  \\
\hline
Число делений шкалы $n$& - & 75 \\
\hline
Цена делений $x_n / n$& - & 2 мА/дел \\
\hline
Чувствительность $ n / x_n$& - & 500 дел/А \\
\hline
Абсолютная погрешность $ \vartriangle x_M$& 1,5 мВ& 0,75 мА\\
\hline
Внутреннее сопротивление прибора (на данном пределе измерений)& 500 Ом & 1 Ом\\
\hline
\end{tabular}
\end{table}
\newpage
\pagebreak 
\centering
\textbf{Результаты измерения вольтамперной характеристики для разных схем и длин проволки:}
\begin{table}
\caption{Результаты а)-схема для $l_1$}
\begin{center}
\begin{tabular}{|r|c|c|c|c|c|c|c|}
\hline
$N_\text{изм}$&1&2&3&4&5&6 \\
\hline
V, мВ&25,1&35,8&45,0&57,7&66,8&74,5 \\
 \hline
 I, мА&48&68&84&112&126&143 \\
 \hline
\end{tabular}
\end{center}

\caption{Результаты а)-схема для $l_2$}
\begin{center}
\begin{tabular}{|r|c|c|c|c|c|c|c|}
\hline
$N_\text{изм}$&1&2&3&4&5&6 \\
\hline
V, мВ&16,8&21,7&27,6&34,1&39,0&45,8\\
 \hline
 I, мА&52&66&88&87&122&142 \\
 \hline
\end{tabular}
\end{center}

\caption{Результаты а)-схема для $l_3$}
\begin{center}
\begin{tabular}{|r|c|c|c|c|c|c|c|}
\hline
$N_\text{изм}$&1&2&3&4&5&6 \\
\hline
V, мВ&11,2&14,2&18,1&22,2&25,9&30,2 \\
 \hline
 I, мА&56&70&86&106&122&144 \\
 \hline
\end{tabular}
\end{center}

\caption{Результаты б)-схема для $l_1$}
\begin{center}
\begin{tabular}{|r|c|c|c|c|c|c|c|}
\hline
$N_\text{изм}$&1&2&3&4&5&6 \\
\hline
V, мВ&25,9&38,0&48,3&58,6&70,0&80,2 \\
 \hline
 I, мА&47&67&85&102&123&145 \\
 \hline
\end{tabular}
\end{center}

\caption{Результаты б)-схема для $l_2$}
\begin{center}
\begin{tabular}{|r|c|c|c|c|c|c|c|}
\hline
$N_\text{изм}$&1&2&3&4&5&6 \\
\hline
V, мВ&16,2&24,0&30,6&38,2&43,0&51,9 \\
 \hline
 I, мА&47&70&85&107&120&149 \\
 \hline
\end{tabular}
\end{center}

\caption{Результаты б)-схема для $l_3$}
\begin{center}
\begin{tabular}{|r|c|c|c|c|c|c|c|}
\hline
$N_\text{изм}$&1&2&3&4&5&6 \\
\hline
V, мВ&11,3&16,0&20,5&24,6&29,7&33,7 \\
 \hline
 I, мА&45&68&85&102&123&140 \\
 \hline
\end{tabular}
\end{center}

\end{table}

\begin{table}
\begin{center}
\caption{Средние величины для а)}
\begin{tabular}{|r|c|c|c|c|}
\hline
&$\left\langle V \right\rangle$ & $\left\langle I \right\rangle$ & $\left\langle I^2 \right\rangle$ & $\left\langle  VI \right\rangle$ \\
\hline
$l_1$&50,8&97&10447&5501 \\
\hline
$l_2$&30,8&96&10169&3299 \\
\hline
$l_3$&20,3&97&10257&2158 \\
\hline
\end{tabular}
\end{center}
\end{table}
\begin{table}
\begin{center}
\caption{Средние величины для б)}
\begin{tabular}{|r|c|c|c|c|}
\hline
&$\left\langle V \right\rangle$ & $\left\langle I \right\rangle$ & $\left\langle I^2 \right\rangle$ & $\left\langle  VI \right\rangle$ \\
\hline
$l_1$&53,5&95&10157&5703 \\
\hline
$l_2$&34,0&96&10385&3672 \\
\hline
$l_3$&22,6&95&9918&2387 \\
\hline
\end{tabular}
\end{center}
\end{table}
\begin{table}
\caption{Результаты измерения сопротивления провoлоки}
\begin{tabular}{|c|c|c|}
\hline
$l_1 = 50,0 \pm 0,1 \text{ см}$&$l_2 = 30,0 \pm 0,1 \text{ см}$&$l_3 = 20,0 \pm 0,1 \text{ см}$ \\
\hline
$R_0$ = 0,5295 $\pm 0,001$ Ом (по Р4833) & $R_0$ = 0,3264  $\pm 0,001$ Ом (по Р4833) & $R_0$ = 0,2166  $\pm 0,001$ Ом (по Р4833) \\
\hline
$\text{Схема а). } R_\text{ср}$ = 0,5274 Ом  & $R_\text{ср}$ = 0,3103 Ом  & $R_\text{ср}$ = 0,2129 Ом  \\
$R_\text{пр}$ = 0,5284 Ом  & $R_\text{пр}$ = 0,3112 Ом  & $R_\text{пр}$ = 0,2135 Ом  \\
$\sigma^{\text{случ}}_{R}$ = 0,007 Ом&$\sigma^{\text{случ}}_{R}$ = 0,004 Ом&  $\sigma^{\text{случ}}_{R}$ = 0,004 Ом \\
$\sigma^{\text{сист}}_{R}$ = 0,005 Ом&$\sigma^{\text{сист}}_{R}$ = 0,003 Ом&  $\sigma^{\text{сист}}_{R}$ = 0,002 Ом \\
$\sigma_{R}$ = 0,009 Ом&$\sigma_{R}$ =  0,005 Ом&$\sigma_{R}$ =   0,004 Ом \\
\hline
$\text{Схема б). } R_\text{ср}$ = 0,5636 Ом  & $R_\text{ср}$ = 0,3575 Ом  & $R_\text{ср}$ = 0,2407 Ом  \\
$R_\text{пр}$ = 0,5629 Ом  & $R_\text{пр}$ = 0,3572 Ом  & $R_\text{пр}$ = 0,2404 Ом  \\
$\sigma^{\text{случ}}_{R}$ = 0,009 Ом&$\sigma^{\text{случ}}_{R}$ = 0,007 Ом&  $\sigma^{\text{случ}}_{R}$ = 0,007 Ом \\
$\sigma^{\text{сист}}_{R}$ = 0,09 Ом&$\sigma^{\text{сист}}_{R}$ = 0,05 Ом&  $\sigma^{\text{сист}}_{R}$ = 0,06 Ом \\
$\sigma_{R}$ = 0,09 Ом&$\sigma_{R}$ =  0,05 Ом&$\sigma_{R}$ =   0,06 Ом \\
\hline
\end{tabular}
\end{table}
\begin{table}
\caption{Расчётные значения удельного сопротивления проволоки}
\centering \begin{tabular}{|c|c|c|}
\hline
l, м& $\rho$, $\dfrac{\text{Ом} \cdot \text{мм}^2}{\text{м}}$& $\sigma_{\rho}$, $\dfrac{\text{Ом} \cdot \text{мм}^2}{\text{м}}$ \\
\hline
0,5&0,97&0,06 \\
0,3&0,98&0,06 \\
0,2&0,98&0,06 \\
\hline
\end{tabular}

\textbf {Литература}\\
Таблицы физических величин: Справочник И.К. Кикоин, 1976 г.
\end{table}
\end{document}